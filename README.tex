% Created 2016-05-04 Wed 00:12
\documentclass[11pt]{article}
\usepackage[utf8]{inputenc}
\usepackage[T1]{fontenc}
\usepackage{fixltx2e}
\usepackage{graphicx}
\usepackage{longtable}
\usepackage{float}
\usepackage{wrapfig}
\usepackage{rotating}
\usepackage[normalem]{ulem}
\usepackage{amsmath}
\usepackage{textcomp}
\usepackage{marvosym}
\usepackage{wasysym}
\usepackage{amssymb}
\usepackage{hyperref}
\tolerance=1000
\author{AnderSon}
\date{Tue May  3 10:45:24 CDT 2016}
\title{\textbf{Iron Brush Tattoo} \emph{Case Jewelry}}
\hypersetup{
  pdfkeywords={},
  pdfsubject={},
  pdfcreator={Emacs 24.5.1 (Org mode 8.2.10)}}
\begin{document}

\maketitle
\tableofcontents

\emph{Rails 4.2.6}

\url{https://github.com/IronBrushTattoo/cj_rails.git}

\section{Config}
\label{sec-1}

\subsection{Gemfile}
\label{sec-1-1}

\url{./Gemfile}

\begin{verbatim}
source 'https://rubygems.org'

gem 'rails', '4.2.6'
gem 'pg', '~> 0.15'
gem 'sass-rails', '~> 5.0'
gem 'uglifier', '>= 1.3.0'
gem 'coffee-rails', '~> 4.1.0'
gem 'jquery-rails'
gem 'turbolinks'
gem 'jbuilder', '~> 2.0'
gem 'sdoc', '~> 0.4.0', group: :doc
gem 'dragonfly', '~> 1.0.12'
gem 'rack-cache', :require => 'rack/cache'

group :development, :test do
  gem 'byebug'
end

group :development do
  gem 'web-console', '~> 2.0'
  gem 'spring'
end
\end{verbatim}

\subsection{Gems}
\label{sec-1-2}

\ref{sec-3-2-1}

\subsection{Environments}
\label{sec-1-3}

\subsubsection{Development}
\label{sec-1-3-1}
\subsubsection{Production}
\label{sec-1-3-2}

\url{./config/environments/production.rb}

\begin{verbatim}
Rails.application.configure do
  # Settings specified here will take precedence over those in config/application.rb.

  # Code is not reloaded between requests.
  config.cache_classes = true

  # Eager load code on boot. This eager loads most of Rails and
  # your application in memory, allowing both threaded web servers
  # and those relying on copy on write to perform better.
  # Rake tasks automatically ignore this option for performance.
  config.eager_load = true

  # Full error reports are disabled and caching is turned on.
  config.consider_all_requests_local       = false
  config.action_controller.perform_caching = true

  # Enable Rack::Cache to put a simple HTTP cache in front of your application
  # Add `rack-cache` to your Gemfile before enabling this.
  # For large-scale production use, consider using a caching reverse proxy like
  # NGINX, varnish or squid.
  config.action_dispatch.rack_cache = true

  # Disable serving static files from the `/public` folder by default since
  # Apache or NGINX already handles this.
  config.serve_static_files = ENV['RAILS_SERVE_STATIC_FILES'].present?

  # Compress JavaScripts and CSS.
  config.assets.js_compressor = :uglifier
  # config.assets.css_compressor = :sass

  # Do not fallback to assets pipeline if a precompiled asset is missed.
  config.assets.compile = false

  # Asset digests allow you to set far-future HTTP expiration dates on all assets,
  # yet still be able to expire them through the digest params.
  config.assets.digest = true

  # `config.assets.precompile` and `config.assets.version` have moved to config/initializers/assets.rb

  # Specifies the header that your server uses for sending files.
  # config.action_dispatch.x_sendfile_header = 'X-Sendfile' # for Apache
  # config.action_dispatch.x_sendfile_header = 'X-Accel-Redirect' # for NGINX

  # Force all access to the app over SSL, use Strict-Transport-Security, and use secure cookies.
  # config.force_ssl = true

  # Use the lowest log level to ensure availability of diagnostic information
  # when problems arise.
  config.log_level = :debug

  # Prepend all log lines with the following tags.
  # config.log_tags = [ :subdomain, :uuid ]

  # Use a different logger for distributed setups.
  # config.logger = ActiveSupport::TaggedLogging.new(SyslogLogger.new)

  # Use a different cache store in production.
  # config.cache_store = :mem_cache_store

  # Enable serving of images, stylesheets, and JavaScripts from an asset server.
  # config.action_controller.asset_host = 'http://assets.example.com'

  # Ignore bad email addresses and do not raise email delivery errors.
  # Set this to true and configure the email server for immediate delivery to raise delivery errors.
  # config.action_mailer.raise_delivery_errors = false

  # Enable locale fallbacks for I18n (makes lookups for any locale fall back to
  # the I18n.default_locale when a translation cannot be found).
  config.i18n.fallbacks = true

  # Send deprecation notices to registered listeners.
  config.active_support.deprecation = :notify

  # Use default logging formatter so that PID and timestamp are not suppressed.
  config.log_formatter = ::Logger::Formatter.new

  # Do not dump schema after migrations.
  config.active_record.dump_schema_after_migration = false
end
\end{verbatim}

\section{First steps}
\label{sec-2}

\begin{verbatim}
rake db:migrate
rake db:setup
\end{verbatim}
\section{Project}
\label{sec-3}

The purpose of this application is to produce several pdf files from an xlsx file,
as a re-implementation of \url{https://github.com/IronBrushTattoo/cj} as a web 
application.

\subsection{User Story}
\label{sec-3-1}

\begin{itemize}
\item user logs in (\ref{sec-3-4})
\begin{itemize}
\item users will be piercers
\end{itemize}
\item chooses xlsx file for upload
\ref{sec-3-2}
\item selects number of days back to make labels from
\item submits
\begin{itemize}
\item BACKGROUND
\begin{itemize}
\item cj-parser.rb does what it does\ldots{}
\begin{itemize}
\item $\square$ rewrite in rails?
\end{itemize}
\end{itemize}
\end{itemize}
\item downloads sheets(pdf files)
\end{itemize}

\subsection{File Upload}
\label{sec-3-2}

\begin{itemize}
\item possible gems
\url{https://www.ruby-toolbox.com/categories/rails_file_uploads}

\begin{itemize}
\item paperclip
\begin{itemize}
\item nb
\begin{itemize}
\item used paperclip before
\item seemed to be designed specifically for image files
\item always worked well
\end{itemize}
\end{itemize}
\item carrierwave
\begin{itemize}
\item nb
\begin{itemize}
\item used before, but not thoroughly
\begin{itemize}
\item i kind of remember having issues with it
\end{itemize}
\end{itemize}
\end{itemize}
\item dragonfly
\url{https://github.com/markevans/dragonfly}
\url{http://markevans.github.io/dragonfly/}
\url{http://markevans.github.io/dragonfly/rails/}

Dragonfly is a highly customizable ruby gem for handling images and other
attachments and is already in use on thousands of websites

\ref{sec-3-2-1}

\begin{itemize}
\item nb
\begin{itemize}
\item used briefly before
\begin{itemize}
\item i remember it being an easy configuration
\end{itemize}
\end{itemize}
\end{itemize}
\item attachment fu
\url{https://github.com/technoweenie/attachment_fu}

Treat an ActiveRecord model as a file attachment, storing its patch, size,
content type, etc. \url{http://weblog.technoweenie.net}

\begin{itemize}
\item nb
\begin{itemize}
\item has not been maintained since Apr 25, 2009
\end{itemize}
\end{itemize}
\item refile
\begin{itemize}
\item nb
\begin{itemize}
\item was my next choice when previously working with file uploads
\end{itemize}
\end{itemize}
\item jquery.fileupload-rails
\item imagery
\item attached
\item papermill
\item fileuploader-rails
\item filecip
\item simple-image-uploader
\end{itemize}
\end{itemize}

\subsubsection{Dragonfly}
\label{sec-3-2-1}

\url{http://markevans.github.io/dragonfly/rails/}

\begin{enumerate}
\item Setup
\label{sec-3-2-1-1}

\begin{itemize}
\item $\boxtimes$ gem 'dragonfly', '\textasciitilde{}> 1.0.12'

\begin{itemize}
\item $\boxtimes$ modify \ref{Gemfile}

\item $\boxtimes$ bundle install
\end{itemize}

\item $\boxtimes$ rails g dragonfly

generates config/initializers/dragonfly.rb

\url{./config/initializers/dragonfly.rb}

\begin{verbatim}
require 'dragonfly'

# Configure
Dragonfly.app.configure do
  plugin :imagemagick

  secret "72245c7371d66ccca6f9356779fa16e3104e6676c1e57af987e9e3f92130dca5"

  url_format "/media/:job/:name"

  datastore :file,
    root_path: Rails.root.join('public/system/dragonfly', Rails.env),
    server_root: Rails.root.join('public')
end

# Logger
Dragonfly.logger = Rails.logger

# Mount as middleware
Rails.application.middleware.use Dragonfly::Middleware

# Add model functionality
if defined?(ActiveRecord::Base)
  ActiveRecord::Base.extend Dragonfly::Model
  ActiveRecord::Base.extend Dragonfly::Model::Validations
end
\end{verbatim}
\end{itemize}

\item Handling attachments
\label{sec-3-2-1-2}

\begin{itemize}
\item example (replace Photo model with Spreadsheet)

Model: \emph{Photo}

\begin{itemize}
\item add \emph{image} attribute to Photo

\begin{verbatim}
class Photo < ActiveRecord::Base
  dragonfly_accessor :image  # defines a reader/writer for image
  # ...
end
\end{verbatim}

\item needs \emph{image$_{\text{uid}}$} column, create migration with

\begin{verbatim}
rails g migration 
\end{verbatim}

\begin{verbatim}
add_column :photos, :image_uid, :string
add_column :photos, :image_name, :string  # Optional - if you want 
                                          # urls to end with the 
                                          # original filename
\end{verbatim}

\item view for uploading

\begin{verbatim}
app/views/photos/...
\end{verbatim}

\begin{verbatim}
<% form_for @photo do |f| %>
  ...
  <%= f.file_field :image %>
  ...
<% end %>
\end{verbatim}

\item allow parameter \emph{image} to be accepted by the controller

\begin{verbatim}
app/controllers/photos_controller.rb
\end{verbatim}

\begin{verbatim}
params.require(:photo).permit(:image)
\end{verbatim}

\item view for displaying

\begin{verbatim}
<%= image_tag @photo.image.thumb('400x200#').url if @photo.image_stored? %>
\end{verbatim}

\item more can be done with \href{http://markevans.github.io/dragonfly/models}{models}
\end{itemize}

\item Spreadsheet model sketch based on above example

Model: \emph{Spreadsheet}

\ref{sec-3-7-2}

\begin{itemize}
\item $\boxtimes$ add \emph{file} attribute to Spreadsheet

\begin{verbatim}
class Spreadsheet < ActiveRecord::Base
  dragonfly_accessor :file  # defines a reader/writer for file
  # ...
end
\end{verbatim}

\item $\boxtimes$ needs \emph{file$_{\text{uid}}$} column, create migration with

\begin{verbatim}
rails g migration AddFileUidToSpreadsheets file_uid:string
rails g migration AddFileNameToSpreadsheets file_name:string
\end{verbatim}

\url{./db/migrate/20160504011342_add_file_uid_to_spreadsheets.rb}
\url{./db/migrate/20160504011542_add_file_name_to_spreadsheets.rb}

\begin{verbatim}
add_column :spreadsheets, :file_uid, :string
add_column :spreadsheets, :file_name, :string  # Optional - if you want 
                                               # urls to end with the 
                                               # original filename
\end{verbatim}

\begin{verbatim}
rake db:migrate
\end{verbatim}

\item $\boxtimes$ view for uploading

\url{./app/views/spreadsheets/}

\begin{verbatim}
<% form_for @spreadsheet do |f| %>
  ...
  <%= f.file_field :file %>
  ...
<% end %>
\end{verbatim}

\item $\boxtimes$ allow parameter \emph{file} to be accepted by the controller

\url{./app/controllers/spreadsheets_controller.rb}

\begin{verbatim}
params.require(:spreadsheet).permit(:file)
\end{verbatim}

\begin{verbatim}
class SpreadsheetsController < ApplicationController
  before_action :set_spreadsheet, only: [:show, :edit, :update, :destroy]

  def index
    @spreadsheets = Spreadsheet.all
  end

  def show
  end

  def new
    @spreadsheet = Spreadsheet.new
  end

  def edit
  end

  def create
    @spreadsheet = Spreadsheet.new(spreadsheet_params)

    respond_to do |format|
      if @spreadsheet.save
        format.html { redirect_to @spreadsheet, notice: 'Spreadsheet was successfully created.' }
        format.json { render :show, status: :created, location: @spreadsheet }
      else
        format.html { render :new }
        format.json { render json: @spreadsheet.errors, status: :unprocessable_entity }
      end
    end
  end

  def update
    respond_to do |format|
      if @spreadsheet.update(spreadsheet_params)
        format.html { redirect_to @spreadsheet, notice: 'Spreadsheet was successfully updated.' }
        format.json { render :show, status: :ok, location: @spreadsheet }
      else
        format.html { render :edit }
        format.json { render json: @spreadsheet.errors, status: :unprocessable_entity }
      end
    end
  end

  def destroy
    @spreadsheet.destroy
    respond_to do |format|
      format.html { redirect_to spreadsheets_url, notice: 'Spreadsheet was successfully destroyed.' }
      format.json { head :no_content }
    end
  end

  private
  def set_spreadsheet
    @spreadsheet = Spreadsheet.find(params[:id])
  end

  def spreadsheet_params
    params.require(:spreadsheet).permit(:index, :file)
  end
end
\end{verbatim}

\item $\boxtimes$ view for displaying

\url{./app/views/spreadsheets/show.html.erb}
\url{./app/views/spreadsheets/index.html.erb}

\begin{verbatim}
<%= @spreadsheet.file_name if @spreadsheet.file_stored? %>
\end{verbatim}

\item more can be done with \href{http://markevans.github.io/dragonfly/models}{models}
\end{itemize}
\end{itemize}

\item Caching
\label{sec-3-2-1-3}

\begin{itemize}
\item $\boxtimes$ \ref{Gemfile}

\begin{verbatim}
gem 'rack-cache', :require => 'rack/cache'
\end{verbatim}

\begin{itemize}
\item $\boxtimes$ bundle install
\end{itemize}

\item $\boxtimes$ uncomment in \ref{sec-1-3-2}

\begin{verbatim}
config.action_dispatch.rack_cache = true
\end{verbatim}
\end{itemize}

\item Custom Endpoints
\label{sec-3-2-1-4}

\ref{sec-3-5-1}

\begin{itemize}
\item $\square$ text generation example

\begin{verbatim}
get "text/:text" => Dragonfly.app.endpoint { |params, app|
  app.generate(:text, params[:text], 'font-size' => 42)
}
\end{verbatim}

\item $\square$ endpoint callable from javascript (e.g. /image?file=egg.png\&size=30x30)

\begin{verbatim}
get "image" => Dragonfly.app.endpoint { |params, app|
  app.fetch_file("some/dir/#{params[:file]}").thumb(params[:size])
}
\end{verbatim}
\end{itemize}
\end{enumerate}

\subsection{File Conversion}
\label{sec-3-3}

\subsubsection{xlsx processing}
\label{sec-3-3-1}

\begin{itemize}
\item roo
\end{itemize}

\subsubsection{latex processing}
\label{sec-3-3-2}

\begin{enumerate}
\item Prawn
\label{sec-3-3-2-1}

\item nb
\label{sec-3-3-2-2}
\url{https://rubygems.org/search?utf8=\%E2\%9C\%93&query=latex}
\url{http://www.sitepoint.com/hackable-pdf-typesetting-in-ruby-with-prawn/}

\url{https://github.com/prawnpdf/prawn}

Prawn is active and looks rad!

\begin{itemize}
\item outdated but possibly useful

\url{https://github.com/baierjan/rails-latex}
\url{https://github.com/bruce/rtex}
\end{itemize}
\end{enumerate}

\subsection{Authentication}
\label{sec-3-4}
\subsection{Views}
\label{sec-3-5}

\subsubsection{Routes}
\label{sec-3-5-1}

\url{./config/routes.rb}

\begin{verbatim}
Rails.application.routes.draw do
  root 'pages#home'

  resources :spreadsheets

  get "spreadsheets" => "spreadsheets#new"

  # The priority is based upon order of creation: first created -> highest priority.
  # See how all your routes lay out with "rake routes".

  # You can have the root of your site routed with "root"
  # root 'welcome#index'

  # Example of regular route:
  #   get 'products/:id' => 'catalog#view'

  # Example of named route that can be invoked with purchase_url(id: product.id)
  #   get 'products/:id/purchase' => 'catalog#purchase', as: :purchase

  # Example resource route (maps HTTP verbs to controller actions automatically):
  #   resources :products

  # Example resource route with options:
  #   resources :products do
  #     member do
  #       get 'short'
  #       post 'toggle'
  #     end
  #
  #     collection do
  #       get 'sold'
  #     end
  #   end

  # Example resource route with sub-resources:
  #   resources :products do
  #     resources :comments, :sales
  #     resource :seller
  #   end

  # Example resource route with more complex sub-resources:
  #   resources :products do
  #     resources :comments
  #     resources :sales do
  #       get 'recent', on: :collection
  #     end
  #   end

  # Example resource route with concerns:
  #   concern :toggleable do
  #     post 'toggle'
  #   end
  #   resources :posts, concerns: :toggleable
  #   resources :photos, concerns: :toggleable

  # Example resource route within a namespace:
  #   namespace :admin do
  #     # Directs /admin/products/* to Admin::ProductsController
  #     # (app/controllers/admin/products_controller.rb)
  #     resources :products
  #   end
end
\end{verbatim}

\ref{sec-3-2-1-4}

\subsubsection{Static Pages}
\label{sec-3-5-2}

\begin{verbatim}
root 'pages#home'    
\end{verbatim}

\ref{sec-3-6-1}

\begin{enumerate}
\item Home
\label{sec-3-5-2-1}

\url{./app/views/pages/home.html.erb}
\end{enumerate}

\subsection{Controllers}
\label{sec-3-6}

\subsubsection{Pages}
\label{sec-3-6-1}

Static pages controller

\begin{verbatim}
rails g controller pages --skip-assets
\end{verbatim}

\subsubsection{Spreadsheets}
\label{sec-3-6-2}

\subsection{Models}
\label{sec-3-7}

\subsubsection{Page}
\label{sec-3-7-1}
\subsubsection{Spreadsheet}
\label{sec-3-7-2}

\url{./app/models/spreadsheet.rb}

\begin{verbatim}
class Spreadsheet < ActiveRecord::Base
  dragonfly_accessor :file  # defines a reader/writer for file
end
\end{verbatim}

\subsection{{\bfseries\sffamily TODO} }
\label{sec-3-8}

\begin{itemize}
\item $\square$ Tests
\item $\square$ sidekiq
\begin{itemize}
\item $\square$ background processes for creating pdfs
\end{itemize}
\item $\square$ requirements
\begin{itemize}
\item $\square$ roo
\item $\square$ chronic
\end{itemize}
\item $\square$ pdflatex
\item $\square$ migrate code from cj-parser
\item $\square$ user authentication
\item $\square$ file upload
\begin{itemize}
\item $\square$ only xlsx file?
\item $\square$ AWS
\end{itemize}
\item $\square$ file storage
\begin{itemize}
\item $\square$ archival api?
\end{itemize}
\end{itemize}
% Emacs 24.5.1 (Org mode 8.2.10)
\end{document}
