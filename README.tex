% Created 2016-05-07 Sat 19:17
\documentclass[11pt]{article}
\usepackage[utf8]{inputenc}
\usepackage[T1]{fontenc}
\usepackage{fixltx2e}
\usepackage{graphicx}
\usepackage{longtable}
\usepackage{float}
\usepackage{wrapfig}
\usepackage{rotating}
\usepackage[normalem]{ulem}
\usepackage{amsmath}
\usepackage{textcomp}
\usepackage{marvosym}
\usepackage{wasysym}
\usepackage{amssymb}
\usepackage{hyperref}
\tolerance=1000
\author{AnderSon}
\date{Tue May  3 10:45:24 CDT 2016}
\title{\textbf{Iron Brush Tattoo} \emph{Case Jewelry}}
\hypersetup{
  pdfkeywords={},
  pdfsubject={},
  pdfcreator={Emacs 24.5.1 (Org mode 8.2.10)}}
\begin{document}

\maketitle
\tableofcontents

\emph{Rails 4.2.6}

\url{https://github.com/IronBrushTattoo/cj_rails.git}
\url{../../Retail_Jewelry}

\section{Config}
\label{sec-1}

\subsection{Gemfile}
\label{sec-1-1}

\url{./Gemfile}

\begin{verbatim}
source 'https://rubygems.org'

gem 'rails', '4.2.6'
gem 'pg', '~> 0.15'
gem 'sass-rails', '~> 5.0'
gem 'uglifier', '>= 1.3.0'
gem 'coffee-rails', '~> 4.1.0'
gem 'jquery-rails'
gem 'turbolinks'
gem 'jbuilder', '~> 2.0'
gem 'sdoc', '~> 0.4.0', group: :doc
gem 'dragonfly', '~> 1.0.12'
gem 'rack-cache', :require => 'rack/cache'
gem 'prawn'
gem 'prawn-table', '~> 0.2.2'
gem 'roo', '~> 2.3.2'
gem 'chronic', '~> 0.10.2'
gem 'omniauth', '~> 1.3'
gem 'omniauth-auth0', '~> 1.4'

group :development, :test do
  gem 'byebug'
end

group :development do
  gem 'web-console', '~> 2.0'
  gem 'spring'
end
\end{verbatim}

\subsection{Gems}
\label{sec-1-2}

\ref{sec-3-2-1}
\ref{sec-3-3-2-1}

\subsection{Environments}
\label{sec-1-3}

\subsubsection{Development}
\label{sec-1-3-1}

\url{./config/environments/development.rb}

\begin{verbatim}
Rails.application.configure do
  # Settings specified here will take precedence over those in config/application.rb.

  # In the development environment your application's code is reloaded on
  # every request. This slows down response time but is perfect for development
  # since you don't have to restart the web server when you make code changes.
  config.cache_classes = false

  # Do not eager load code on boot.
  config.eager_load = false

  # Show full error reports and disable caching.
  config.consider_all_requests_local       = true
  config.action_controller.perform_caching = false

  # Don't care if the mailer can't send.
  config.action_mailer.raise_delivery_errors = false

  # Print deprecation notices to the Rails logger.
  config.active_support.deprecation = :log

  # Raise an error on page load if there are pending migrations.
  config.active_record.migration_error = :page_load

  # Debug mode disables concatenation and preprocessing of assets.
  # This option may cause significant delays in view rendering with a large
  # number of complex assets.
  config.assets.debug = true

  # Asset digests allow you to set far-future HTTP expiration dates on all assets,
  # yet still be able to expire them through the digest params.
  config.assets.digest = true

  # Adds additional error checking when serving assets at runtime.
  # Checks for improperly declared sprockets dependencies.
  # Raises helpful error messages.
  config.assets.raise_runtime_errors = true

  # Raises error for missing translations
  # config.action_view.raise_on_missing_translations = true

  OmniAuth.config.on_failure = Proc.new { |env|
    message_key = env['omniauth.error.type']
    error_description = Rack::Utils.escape(env['omniauth.error'].error_reason)
    new_path = "#{env['SCRIPT_NAME']}#{OmniAuth.config.path_prefix}/failure?message=#{message_key}&error_description=#{error_description}"
    Rack::Response.new(['302 Moved'], 302, 'Location' => new_path).finish
  }
end
\end{verbatim}

\subsubsection{Production}
\label{sec-1-3-2}

\url{./config/environments/production.rb}

\begin{verbatim}
Rails.application.configure do
  # Settings specified here will take precedence over those in config/application.rb.

  # Code is not reloaded between requests.
  config.cache_classes = true

  # Eager load code on boot. This eager loads most of Rails and
  # your application in memory, allowing both threaded web servers
  # and those relying on copy on write to perform better.
  # Rake tasks automatically ignore this option for performance.
  config.eager_load = true

  # Full error reports are disabled and caching is turned on.
  config.consider_all_requests_local       = false
  config.action_controller.perform_caching = true

  # Enable Rack::Cache to put a simple HTTP cache in front of your application
  # Add `rack-cache` to your Gemfile before enabling this.
  # For large-scale production use, consider using a caching reverse proxy like
  # NGINX, varnish or squid.
  config.action_dispatch.rack_cache = true

  # Disable serving static files from the `/public` folder by default since
  # Apache or NGINX already handles this.
  config.serve_static_files = ENV['RAILS_SERVE_STATIC_FILES'].present?

  # Compress JavaScripts and CSS.
  config.assets.js_compressor = :uglifier
  # config.assets.css_compressor = :sass

  # Do not fallback to assets pipeline if a precompiled asset is missed.
  config.assets.compile = false

  # Asset digests allow you to set far-future HTTP expiration dates on all assets,
  # yet still be able to expire them through the digest params.
  config.assets.digest = true

  # `config.assets.precompile` and `config.assets.version` have moved to config/initializers/assets.rb

  # Specifies the header that your server uses for sending files.
  # config.action_dispatch.x_sendfile_header = 'X-Sendfile' # for Apache
  # config.action_dispatch.x_sendfile_header = 'X-Accel-Redirect' # for NGINX

  # Force all access to the app over SSL, use Strict-Transport-Security, and use secure cookies.
  # config.force_ssl = true

  # Use the lowest log level to ensure availability of diagnostic information
  # when problems arise.
  config.log_level = :debug

  # Prepend all log lines with the following tags.
  # config.log_tags = [ :subdomain, :uuid ]

  # Use a different logger for distributed setups.
  # config.logger = ActiveSupport::TaggedLogging.new(SyslogLogger.new)

  # Use a different cache store in production.
  # config.cache_store = :mem_cache_store

  # Enable serving of images, stylesheets, and JavaScripts from an asset server.
  # config.action_controller.asset_host = 'http://assets.example.com'

  # Ignore bad email addresses and do not raise email delivery errors.
  # Set this to true and configure the email server for immediate delivery to raise delivery errors.
  # config.action_mailer.raise_delivery_errors = false

  # Enable locale fallbacks for I18n (makes lookups for any locale fall back to
  # the I18n.default_locale when a translation cannot be found).
  config.i18n.fallbacks = true

  # Send deprecation notices to registered listeners.
  config.active_support.deprecation = :notify

  # Use default logging formatter so that PID and timestamp are not suppressed.
  config.log_formatter = ::Logger::Formatter.new

  # Do not dump schema after migrations.
  config.active_record.dump_schema_after_migration = false

  OmniAuth.config.on_failure = Proc.new { |env|
    message_key = env['omniauth.error.type']
    error_description = Rack::Utils.escape(env['omniauth.error'].error_reason)
    new_path = "#{env['SCRIPT_NAME']}#{OmniAuth.config.path_prefix}/failure?message=#{message_key}&error_description=#{error_description}"
    Rack::Response.new(['302 Moved'], 302, 'Location' => new_path).finish
  }

end
\end{verbatim}

\ref{sec-3-4-1}

\section{First steps}
\label{sec-2}

\begin{verbatim}
rake db:migrate
rake db:setup
\end{verbatim}
\section{Project}
\label{sec-3}

The purpose of this application is to produce several pdf files from an xlsx file,
as a re-implementation of \url{https://github.com/IronBrushTattoo/cj} as a web 
application.

\subsection{User Story}
\label{sec-3-1}

\begin{itemize}
\item user logs in (\ref{sec-3-4})
\item chooses xlsx file for upload
\ref{sec-3-2}
\item selects number of days back to make labels from
\item submits
\begin{itemize}
\item BACKGROUND
\begin{itemize}
\item cj-parser.rb does what it does\ldots{}
\begin{itemize}
\item $\square$ rewrite in rails?
\end{itemize}
\end{itemize}
\end{itemize}
\item downloads sheets(pdf files)
\end{itemize}

\subsection{File Upload}
\label{sec-3-2}

\begin{itemize}
\item possible gems
\url{https://www.ruby-toolbox.com/categories/rails_file_uploads}

\begin{itemize}
\item paperclip
\begin{itemize}
\item nb
\begin{itemize}
\item used paperclip before
\item seemed to be designed specifically for image files
\item always worked well
\end{itemize}
\end{itemize}
\item carrierwave
\begin{itemize}
\item nb
\begin{itemize}
\item used before, but not thoroughly
\begin{itemize}
\item i kind of remember having issues with it
\end{itemize}
\end{itemize}
\end{itemize}
\item dragonfly
\url{https://github.com/markevans/dragonfly}
\url{http://markevans.github.io/dragonfly/}
\url{http://markevans.github.io/dragonfly/rails/}

Dragonfly is a highly customizable ruby gem for handling images and other
attachments and is already in use on thousands of websites

\ref{sec-3-2-1}

\begin{itemize}
\item nb
\begin{itemize}
\item used briefly before
\begin{itemize}
\item i remember it being an easy configuration
\end{itemize}
\end{itemize}
\end{itemize}
\item attachment fu
\url{https://github.com/technoweenie/attachment_fu}

Treat an ActiveRecord model as a file attachment, storing its patch, size,
content type, etc. \url{http://weblog.technoweenie.net}

\begin{itemize}
\item nb
\begin{itemize}
\item has not been maintained since Apr 25, 2009
\end{itemize}
\end{itemize}
\item refile
\begin{itemize}
\item nb
\begin{itemize}
\item was my next choice when previously working with file uploads
\end{itemize}
\end{itemize}
\item jquery.fileupload-rails
\item imagery
\item attached
\item papermill
\item fileuploader-rails
\item filecip
\item simple-image-uploader
\end{itemize}
\end{itemize}

\subsubsection{Dragonfly}
\label{sec-3-2-1}

\url{http://markevans.github.io/dragonfly/rails/}

\begin{enumerate}
\item Setup
\label{sec-3-2-1-1}

\begin{itemize}
\item $\boxtimes$ gem 'dragonfly', '\textasciitilde{}> 1.0.12'

\begin{itemize}
\item $\boxtimes$ modify \ref{Gemfile}

\item $\boxtimes$ bundle install
\end{itemize}

\item $\boxtimes$ rails g dragonfly

generates config/initializers/dragonfly.rb

\url{./config/initializers/dragonfly.rb}

\begin{verbatim}
require 'dragonfly'

# Configure
Dragonfly.app.configure do
  plugin :imagemagick

  secret "72245c7371d66ccca6f9356779fa16e3104e6676c1e57af987e9e3f92130dca5"

  url_format "/media/:job/:name"

  datastore :file,
    root_path: Rails.root.join('public/system/dragonfly', Rails.env),
    server_root: Rails.root.join('public')
end

# Logger
Dragonfly.logger = Rails.logger

# Mount as middleware
Rails.application.middleware.use Dragonfly::Middleware

# Add model functionality
if defined?(ActiveRecord::Base)
  ActiveRecord::Base.extend Dragonfly::Model
  ActiveRecord::Base.extend Dragonfly::Model::Validations
end
\end{verbatim}
\end{itemize}

\item Handling attachments
\label{sec-3-2-1-2}

\begin{itemize}
\item example (replace Photo model with Spreadsheet)

Model: \emph{Photo}

\begin{itemize}
\item add \emph{image} attribute to Photo

\begin{verbatim}
class Photo < ActiveRecord::Base
  dragonfly_accessor :image  # defines a reader/writer for image
  # ...
end
\end{verbatim}

\item needs \emph{image$_{\text{uid}}$} column, create migration with

\begin{verbatim}
rails g migration 
\end{verbatim}

\begin{verbatim}
add_column :photos, :image_uid, :string
add_column :photos, :image_name, :string  # Optional - if you want 
                                          # urls to end with the 
                                          # original filename
\end{verbatim}

\item view for uploading

\begin{verbatim}
app/views/photos/...
\end{verbatim}

\begin{verbatim}
<% form_for @photo do |f| %>
  ...
  <%= f.file_field :image %>
  ...
<% end %>
\end{verbatim}

\item allow parameter \emph{image} to be accepted by the controller

\begin{verbatim}
app/controllers/photos_controller.rb
\end{verbatim}

\begin{verbatim}
params.require(:photo).permit(:image)
\end{verbatim}

\item view for displaying

\begin{verbatim}
<%= image_tag @photo.image.thumb('400x200#').url if @photo.image_stored? %>
\end{verbatim}

\item more can be done with \href{http://markevans.github.io/dragonfly/models}{models}
\end{itemize}

\item Spreadsheet model sketch based on above example

Model: \emph{Spreadsheet}

\ref{sec-3-7-2}

\begin{itemize}
\item $\boxtimes$ add \emph{file} attribute to Spreadsheet

\begin{verbatim}
class Spreadsheet < ActiveRecord::Base
  dragonfly_accessor :file  # defines a reader/writer for file
  # ...
end
\end{verbatim}

\item $\boxtimes$ needs \emph{file$_{\text{uid}}$} column, create migration with

\begin{verbatim}
rails g migration AddFileUidToSpreadsheets file_uid:string
rails g migration AddFileNameToSpreadsheets file_name:string
\end{verbatim}

\url{./db/migrate/20160504011342_add_file_uid_to_spreadsheets.rb}
\url{./db/migrate/20160504011542_add_file_name_to_spreadsheets.rb}

\begin{verbatim}
add_column :spreadsheets, :file_uid, :string
add_column :spreadsheets, :file_name, :string  # Optional - if you want 
                                               # urls to end with the 
                                               # original filename
\end{verbatim}

\begin{verbatim}
rake db:migrate
\end{verbatim}

\item $\boxtimes$ view for uploading

\url{./app/views/spreadsheets/}

\begin{verbatim}
<% form_for @spreadsheet do |f| %>
  ...
  <%= f.file_field :file %>
  ...
<% end %>
\end{verbatim}

\item $\boxtimes$ allow parameter \emph{file} to be accepted by the controller

\url{./app/controllers/spreadsheets_controller.rb}

\begin{verbatim}
params.require(:spreadsheet).permit(:file)
\end{verbatim}

\ref{SpreadsheetPdf}

\begin{verbatim}
class SpreadsheetsController < ApplicationController
  before_action :set_spreadsheet, only: [:show, :edit, :update, :destroy]

  def index
    @spreadsheets = Spreadsheet.all
  end

  def show
    @spreadsheet = Spreadsheet.find(params[:id])
    respond_to do |format|
      format.html
      format.pdf do
        pdf = SpreadsheetPdf.new(@spreadsheet, view_context)
        send_data pdf.render,
                  filename: "spreadsheet_#{@spreadsheet.created_at.strftime("%d/%m/%Y")}.pdf",
                  type: "application/pdf",
                  disposition: "inline"
      end
    end
  end

  def new
    @spreadsheet = Spreadsheet.new
  end

  def edit
  end

  def create
    @spreadsheet = Spreadsheet.new(spreadsheet_params)

    respond_to do |format|
      if @spreadsheet.save
        format.html { redirect_to @spreadsheet, notice: 'Spreadsheet was successfully created.' }
        format.json { render :show, status: :created, location: @spreadsheet }
      else
        format.html { render :new }
        format.json { render json: @spreadsheet.errors, status: :unprocessable_entity }
      end
    end
  end

  def update
    respond_to do |format|
      if @spreadsheet.update(spreadsheet_params)
        format.html { redirect_to @spreadsheet, notice: 'Spreadsheet was successfully updated.' }
        format.json { render :show, status: :ok, location: @spreadsheet }
      else
        format.html { render :edit }
        format.json { render json: @spreadsheet.errors, status: :unprocessable_entity }
      end
    end
  end

  def destroy
    @spreadsheet.destroy
    respond_to do |format|
      format.html { redirect_to spreadsheets_url, notice: 'Spreadsheet was successfully destroyed.' }
      format.json { head :no_content }
    end
  end

  private
  def set_spreadsheet
    @spreadsheet = Spreadsheet.find(params[:id])
  end

  def spreadsheet_params
    params.require(:spreadsheet).permit(:index, :file, :days)
  end
end
\end{verbatim}

\begin{itemize}
\item nb

\ref{sec-3-3-2-1}
\ref{sec-3-3-2-1-2-1}
\end{itemize}

\item $\boxtimes$ view for displaying

\url{./app/views/spreadsheets/show.html.erb}
\url{./app/views/spreadsheets/index.html.erb}

\begin{verbatim}
<%= @spreadsheet.file_name if @spreadsheet.file_stored? %>
\end{verbatim}

\item more can be done with \href{http://markevans.github.io/dragonfly/models}{models}
\end{itemize}
\end{itemize}

\item Caching
\label{sec-3-2-1-3}

\begin{itemize}
\item $\boxtimes$ \ref{Gemfile}

\begin{verbatim}
gem 'rack-cache', :require => 'rack/cache'
\end{verbatim}

\begin{itemize}
\item $\boxtimes$ bundle install
\end{itemize}

\item $\boxtimes$ uncomment in \ref{sec-1-3-2}

\begin{verbatim}
config.action_dispatch.rack_cache = true
\end{verbatim}
\end{itemize}

\item Custom Endpoints
\label{sec-3-2-1-4}

\ref{sec-3-5-1}

\begin{itemize}
\item $\square$ text generation example

\begin{verbatim}
get "text/:text" => Dragonfly.app.endpoint { |params, app|
  app.generate(:text, params[:text], 'font-size' => 42)
}
\end{verbatim}

\item $\square$ endpoint callable from javascript (e.g. /image?file=egg.png\&size=30x30)

\begin{verbatim}
get "image" => Dragonfly.app.endpoint { |params, app|
  app.fetch_file("some/dir/#{params[:file]}").thumb(params[:size])
}
\end{verbatim}
\end{itemize}

\item Using Another Data Store
\label{sec-3-2-1-5}

\url{http://www.sitepoint.com/file-uploads-dragonfly/}

\begin{itemize}
\item $\square$ add \url{./public/system/dragonfly} to \url{./.gitignore}

\begin{verbatim}
# See https://help.github.com/articles/ignoring-files for more about ignoring files.
#
# If you find yourself ignoring temporary files generated by your text editor
# or operating system, you probably want to add a global ignore instead:
#   git config --global core.excludesfile '~/.gitignore_global'

# Ignore bundler config.
/.bundle

# Ignore all logfiles and tempfiles.
/log/*
!/log/.keep
/tmp

/public/system/dragonfly
\end{verbatim}
\end{itemize}
\end{enumerate}

\subsection{File Conversion}
\label{sec-3-3}

\subsubsection{xlsx processing}
\label{sec-3-3-1}

\begin{enumerate}
\item Roo
\label{sec-3-3-1-1}

\url{https://github.com/roo-rb/roo}

\begin{itemize}
\item $\boxtimes$ \ref{Gemfile}

\begin{verbatim}
gem 'roo', '~> 2.3.2'
\end{verbatim}

\begin{itemize}
\item $\boxtimes$ bundle install
\end{itemize}

\item $\square$ examine code from native application

\url{file:///home/son/IBT/jewelry/Retail_Jewelry/cj-parser.rb}

\begin{verbatim}
: require 'roo'
: 
: def get_labels(file)
:   puts "getting labels"
:   
:   labels = []
:   
:   xls_file = Roo::Spreadsheet.open(file)
: 
:   xls_file.sheets.each do |sheet|
: 
:     sheet = xls_file.sheet(sheet)
:     
:     sheet.parse[4..-1].each do |row|
: 
:       zero,one,two,four,five,ten = nil_convert(row[0]),
:       nil_convert(row[1]),
:       nil_convert(row[2]),
:       nil_convert(row[4]),
:       nil_convert(row[5]),
:       nil_convert(row[10])
: 
:       sizes = strip(five.to_s)
:       gauge = "#{sizes[0]}g"
:       size = "#{sizes[1]}\""
:       desc = two.gsub("&", "and")
:       id = one.to_s.split(/-/)[0]
:       price = "$#{four.to_s.split(".")[0]}"
:       supply = five
:       updated = Chronic.parse(ten).to_f
: 
:       label = Label.new(gauge,
:                         size,
:                         desc,
:                         id,
:                         price,
:                         supply,
:                         updated
:                        )
: 
:       seconds = 60*60*24*$days
:       
:       if (Time.now.to_f - updated.to_f) < seconds
:         puts label.id
:         $labelID = label.id
:         labels.push label
:       end
: 
:     end
:   end
: 
:   return labels
: 
: end
\end{verbatim}
\end{itemize}
\end{enumerate}

\subsubsection{pdf processing}
\label{sec-3-3-2}

\begin{enumerate}
\item Prawn
\label{sec-3-3-2-1}
\url{https://github.com/prawnpdf/prawn}

\begin{enumerate}
\item Setup
\label{sec-3-3-2-1-1}

\begin{itemize}
\item $\boxtimes$ \ref{Gemfile}
\end{itemize}
\begin{verbatim}
#gem 'prawn', '~> 2.1'
gem 'prawn'
\end{verbatim}

\begin{itemize}
\item $\boxtimes$ bundle install
\end{itemize}

\begin{enumerate}
\item Manual
\label{sec-3-3-2-1-1-1}

\url{http://prawnpdf.org/manual.pdf}

\begin{itemize}
\item $\boxtimes$ clone repository

\begin{verbatim}
git clone https://github.com/prawnpdf/prawn
\end{verbatim}

\item $\square$ switch to the stable branch 

\begin{verbatim}
git branch stable
\end{verbatim}

\item $\boxtimes$ bundle install

\item $\square$ bundle exec rake manual
generates \emph{manual.pdf} in the project root
\end{itemize}
\end{enumerate}

\item Tutorials
\label{sec-3-3-2-1-2}
\begin{enumerate}
\item Creating PDF Using Prawn in Ruby on Rails
\label{sec-3-3-2-1-2-1}

\url{http://www.idyllic-software.com/blog/creating-pdf-using-prawn-in-ruby-on-rails/}

\begin{itemize}
\item $\boxtimes$ \ref{Gemfile}

\begin{verbatim}
gem 'prawn'
bundle install
\end{verbatim}

\item $\boxtimes$ \url{./config/initializers/mime_types.rb}

create a PDF Mime::Type inside mime$_{\text{types}}$.rb

\begin{verbatim}
# Be sure to restart your server when you modify this file.

# Add new mime types for use in respond_to blocks:
# Mime::Type.register "text/richtext", :rtf

Mime::Type.register "application/pdf", :pdf
\end{verbatim}

\item $\boxtimes$ \url{./app/controllers/spreadsheets_controller.rb}

\texttt{Spreadsheets}

\url{http://www.idyllic-software.com/blog/creating-pdf-using-prawn-in-ruby-on-rails/}

\begin{verbatim}
class InvoicesController < ApplicationController

  before_filter :authenticate_customer!, :only => [:index, :show]

  def index
    @invoices = Invoice.all_invoices(current_customer)
  end

  def show
    @invoice = Invoice.find(params[:id])
    respond_to do |format|
      format.html
      format.pdf do
        pdf = InvoicePdf.new(@invoice, view_context)
        send_data pdf.render, filename: 
          "invoice_#{@invoice.created_at.strftime("%d/%m/%Y")}.pdf",
          type: "application/pdf"
      end
    end
  end

end
\end{verbatim}

\begin{itemize}
\item $\boxtimes$ open pdf in browser instead of download

add /disposition: "inline"/ after type
\end{itemize}

\item $\boxtimes$ create a new class \textbf{app/pdfs/spreadsheet$_{\text{pdf}}$}

\ref{SpreadsheetPdf}

\texttt{Spreadsheets}

\item $\boxtimes$ restart server

\item $\boxtimes$ \ref{SpreadsheetPdf}

set the \emph{@spreadsheet} and \emph{view$_{\text{context}}$}

\begin{verbatim}
def initialize(spreadsheet, view)
  super()
  @spreadsheet = spreadsheet
  @view = view
  text "Spreadsheet #{@spreadsheet.id}"
end
\end{verbatim}

\item $\square$ create different methods inside \ref{SpreadsheetPdf} as per what you want to show on the pdf

\begin{itemize}
\item example

\begin{verbatim}
def logo
  logopath = "#{Rails.root}/app/assets/images/logo.png"
  image logopath, :width => 197, :height => 91
end
\end{verbatim}
\end{itemize}
\end{itemize}

\begin{enumerate}
\item Issues
\label{sec-3-3-2-1-2-1-1}

\begin{itemize}
\item $\square$ NameError (uninitialized constant SpreadsheetsController::SpreadsheetPdf)
\end{itemize}
\end{enumerate}
\end{enumerate}
\end{enumerate}

\item nb
\label{sec-3-3-2-2}
\url{https://rubygems.org/search?utf8=\%E2\%9C\%93&query=latex}
\url{http://www.sitepoint.com/hackable-pdf-typesetting-in-ruby-with-prawn/}

\url{https://github.com/prawnpdf/prawn}

\ref{sec-3-3-2-1} is active and looks rad!

\begin{itemize}
\item outdated but possibly useful

\url{https://github.com/baierjan/rails-latex}
\url{https://github.com/bruce/rtex}
\end{itemize}
\end{enumerate}

\subsection{Authentication}
\label{sec-3-4}

\subsubsection{AuthO}
\label{sec-3-4-1}

\url{https://manage.auth0.com/#/}
\url{https://manage.auth0.com/#/applications/eD2I5SEBmvgx3tEzeFd35mJfvW5HkasK/quickstart}

\begin{itemize}
\item $\boxtimes$ 1. Add dependencies

\ref{Gemfile}

\begin{verbatim}
gem 'omniauth', '~> 1.3'
gem 'omniauth-auth0', '~> 1.4'
\end{verbatim}

\begin{itemize}
\item $\boxtimes$ bundle install
\end{itemize}

\item $\square$ 2. Initialize Omniauth AuthO

\url{./config/initializers/auth0.rb}

\begin{verbatim}
Rails.application.config.middleware.use OmniAuth::Builder do
  provider(
    :auth0,
    'eD2I5SEBmvgx3tEzeFd35mJfvW5HkasK',
    'CbeRu17uzZttvi8qLafpZJLWr2vTcEf33VLAztuT16FmeujTgrHLtWMSxSe-Plwf',
    'sonarch.auth0.com',
    callback_path: "/auth/auth0/callback"
  )
end
\end{verbatim}

\begin{itemize}
\item $\square$ add to gitignore?
\end{itemize}

\item $\boxtimes$ 3. Add the AuthO callback handler

\begin{verbatim}
rails g controller auth0 callback failure --skip-template-engine --skip-assets
\end{verbatim}

\url{./app/controllers/auth0_controller.rb}

\begin{verbatim}
class Auth0Controller < ApplicationController
  def callback
    # This stores all the user information that came from Auth0 and the IdP
    session[:userinfo] = request.env['ominauth.auth']

    # Redirect to the URL you want after successful auth
    redirect_to '/dashboard'
  end

  def failure
    # show a failure page or redirect to an error page
    @error_msg = request.params['message']
  end
end
\end{verbatim}

\begin{itemize}
\item $\boxtimes$ replace the generated routes in \url{./config/routes.rb} 

\ref{sec-3-5-1}

\begin{verbatim}
get "/auth/auth0/callback" => "auth0#callback"
get "/auth/failure" => "auth0#failure"
\end{verbatim}
\end{itemize}

\item $\boxtimes$ 4. Specify the callback on AuthO Dashboard

For security purposes, register callback URL of the application at
\url{https://manage.auth0.com/#/applications} 

\begin{verbatim}
http://yourUrl/auth/auth0/callback
\end{verbatim}

\item $\boxtimes$ 5. Triggering login manually or integrating the AuthOLock

\begin{itemize}
\item $\boxtimes$ passwordless - Email Code

\url{./app/views/pages/login.html.erb}

\begin{verbatim}
<script src="https://cdn.auth0.com/js/lock-passwordless-1.0.min.js"></script>
<script type="text/javascript">
  var clientID = 'eD2I5SEBmvgx3tEzeFd35mJfvW5HkasK';
  var domain = 'sonarch.auth0.com';

  var lock = new Auth0LockPasswordless(clientID, domain);

  function open() {
  lock.emailcode({
  callbackURL: 'http://a1e4c04e.ngrok.io/auth/auth0/callback',
  authParams: {
  scope: 'openid profile'
  }
  });
  }
</script>
<button onclick="window.open();">Email Code</button>
\end{verbatim}

update \ref{sec-3-5-1}
\end{itemize}

\item $\square$ 6. Accessing user information

via \emph{userinfo} stored in the session on step 3

\begin{verbatim}
class DashboardController < SecuredController
  def show
    @user = session[:userinfo]
  end
end
\end{verbatim}

\begin{verbatim}
<div>
  <img class="avatar" src="<%= @user[:info][:image] %>"/>
  <h2>Welcom <%= @user[:info][:name] %></h2>
</div>
\end{verbatim}

\url{./app/views/dashboard/show.html.erb}

\begin{verbatim}
<div>
  <img class="avatar" src="<%= @user[:info][:image] %>"/>
  <h2>Welcom <%= @user[:info][:name] %></h2>
</div>
\end{verbatim}

\ref{sec-3-6-2}
\ref{sec-3-6-1}

\item $\square$ Troubleshooting

\begin{itemize}
\item $\square$ "We're sorry, we can't send you the email\ldots{}

\href{https://www.google.com/search?q=auth0+in+development&oq=auth0+in+development&aqs=chrome..69i57.6898j0j7&sourceid=chrome&es_sm=122&ie=UTF-8}{auth0 in development}

\url{https://auth0.com/docs/lifecycle}
\end{itemize}
\end{itemize}


\begin{itemize}
\item $\square$ get error description on failure

\url{./config/environments/production.rb}

\begin{verbatim}
OmniAuth.config.on_failure = Proc.new { |env|
  message_key = env['omniauth.error.type']
  error_description = Rack::Utils.escape(env['omniauth.error'].error_reason)
  new_path = "#{env['SCRIPT_NAME']}#{OmniAuth.config.path_prefix}/failure?message=#{message_key}&error_description=#{error_description}"
  Rack::Response.new(['302 Moved'], 302, 'Location' => new_path).finish
}
\end{verbatim}

\ref{sec-1}
\ref{sec-1-3}
\ref{sec-1-3-2}
\end{itemize}

\subsubsection{nb}
\label{sec-3-4-2}

\begin{itemize}
\item oauth?
\begin{itemize}
\item login with familiar accounts
\begin{itemize}
\item google, fb, twitter, etc
\end{itemize}
\end{itemize}
\end{itemize}

\subsection{Views}
\label{sec-3-5}

\subsubsection{Routes}
\label{sec-3-5-1}

\url{./config/routes.rb}

\begin{verbatim}
Rails.application.routes.draw do
  root 'spreadsheets#new'

  resources :spreadsheets

  get "spreadsheets" => "spreadsheets#new"
  get "login" => "pages#login"

  get "/auth/auth0/callback" => "auth0#callback"
  get "/auth/failure" => "auth0#failure"
end
\end{verbatim}

\ref{sec-3-6-1}
\ref{sec-3-4-1}
\ref{sec-3-2-1-4}

\subsubsection{Static Pages}
\label{sec-3-5-2}

\begin{verbatim}
root 'pages#home'    
\end{verbatim}

\ref{sec-3-6-1}

\begin{enumerate}
\item Home
\label{sec-3-5-2-1}

\url{./app/views/pages/home.html.erb}
\end{enumerate}

\subsection{Controllers}
\label{sec-3-6}

\subsubsection{Pages}
\label{sec-3-6-1}

Static pages controller

\begin{verbatim}
rails g controller pages --skip-assets
\end{verbatim}

\url{./app/controllers/pages_controller.rb}

\begin{verbatim}
class PagesController < ApplicationController
end
\end{verbatim}

\begin{itemize}
\item nb

\ref{sec-3-4-1}
\end{itemize}

\subsubsection{Dashboard}
\label{sec-3-6-2}

\begin{verbatim}
rails g controller Dashboard show
\end{verbatim}

\url{./app/controllers/dashboard_controller.rb}

\begin{verbatim}
class DashboardController < SecuredController
  def show
    @user = session[:userinfo]
  end
end
\end{verbatim}

\subsection{Models}
\label{sec-3-7}

\subsubsection{Page}
\label{sec-3-7-1}
\subsubsection{Spreadsheet}
\label{sec-3-7-2}

\url{./app/models/spreadsheet.rb}

\begin{verbatim}
class Spreadsheet < ActiveRecord::Base
  dragonfly_accessor :file  # defines a reader/writer for file
end
\end{verbatim}

\subsubsection{Label}
\label{sec-3-7-3}
\subsection{Classes}
\label{sec-3-8}

\subsubsection{SpreadsheetPdf}
\label{sec-3-8-1}

\begin{verbatim}
mkdir ./app/pdfs
\end{verbatim}

\url{./app/pdfs/spreadsheet_pdf.rb}

\begin{verbatim}
class SpreadsheetPdf < Prawn::Document

  def initialize(spreadsheet, view)
    super()
    @spreadsheet = spreadsheet
    file_path = "public/system/dragonfly/development"
    xls_file = get_labels("#{file_path}/#{@spreadsheet.file_uid}")
    @view = view

    make_labels(xls_file)

  end

  def desc_box(desc)
    formatted_text_box [
      {
        text: "#{desc}",
        color: "ffffff",
        styles: [:italic],
        size: 11
      }
    ], at: [$padding,$top], width: $text_width
  end

  def size_box(gauge,size)
    transparent(0.1) do
      stroke_rectangle [0, $box_height+$padding], $box_width, $box_height*0.75
    end

    transparent(0.6) do
      formatted_text_box [
        {
          text: "#{gauge} #{size}",
          color: "ffffff",
          size: 15,
          styles: [:normal],
          character_spacing: 0.5
        }
      ], at: [($box_width/2)-$padding,$mid], width: $text_width
    end

  end

  def id_box(id)
    formatted_text_box [
      {
        text: "#{id}",        
        color: "ffffff",
        align: :left
      }
    ], at: [$padding,$low], width: $text_width
  end

  def price_box(price)
    formatted_text_box [
      {
        text: "#{price}",
        color: "ffffff",
        align: :right,
        styles: [:italic]
      }
    ], at: [$box_width-($padding*3),$low], width: $text_width
  end

  def make_labels(file)

    define_grid(:columns => 4, :rows => 8, :row_gutter => 10, :column_gutter => 0)
    margin = 14

    count = 1
    row = 0
    column = 0

    start_new_page(:margin => margin)

    file.each do |label|

      grid(row, column).bounding_box do
        #stroke_axis

        fill_color "000000"
        stroke_color "ffffff"

        $box_width = 144
        $box_height = 81
        $padding = 12
        $text_width = 90
        $top = $box_height-$padding
        $mid = $top - 30
        $low = $mid - 20

        transparent(0.9) do
          fill_rectangle [0, $box_height], $box_width, $box_height

          font "Times-Roman", :size => 10

          desc_box(label.desc)
          size_box(label.gauge, label.size)
          id_box(label.id)
          price_box(label.price)
        end

      end

      if count%4 == 0
        row+=1
        column = 0
      else
        column+=1
      end

      if count%32 == 0
        start_new_page(:margin => margin)
        row = 0
        column = 0
      end

      count+=1
    end

  end

  def logo
    logopath = "#{Rails.root}/app/assets/images/logo.png"
    image logopath, :width => 197, :height => 197
  end

  def get_labels(file)

    labels = []

    xls_file = Roo::Spreadsheet.open(file, extension: :xlsx)

    xls_file.sheets.each do |sheet|

      sheet = xls_file.sheet(sheet)

      sheet.parse[0..-1].each do |row|

        zero,one,two,four,five,ten = nil_convert(row[0]),
        nil_convert(row[1]),
        nil_convert(row[2]),
        nil_convert(row[4]),
        nil_convert(row[5]),
        nil_convert(row[10])

        sizes = strip(five.to_s)
        gauge = "#{sizes[0]}g" unless sizes[0].nil?
        size = "#{sizes[1]}\"" unless sizes[1].nil?
        desc = two.gsub("&", "and")
        id = one.to_s.split(/-/)[0]
        price = "$#{four.to_s.split(".")[0]}"
        supply = five
        updated = Chronic.parse(ten).to_f

        label = Label.new(gauge,
                          size,
                          desc,
                          id,
                          price,
                          supply,
                          updated
                         )

        seconds = 60*60*24*@spreadsheet.days.to_f

        if (Time.now.to_f - updated.to_f) < seconds
          #puts label.id
          $labelID = label.id
          labels.push label
        end

      end
    end

    return labels

  end

  def nil_convert(c)
    if c.nil? 
      ""
    else
      c
    end
  end

  def strip(s)
    s.gsub(/"/, '').
      gsub(/g/, '').
      gsub(/G/, '').
      gsub(/,/, '').
      split(' ')
  end

end
\end{verbatim}

\begin{itemize}
\item nb

\ref{sec-3-3-2-1}

\texttt{Spreadsheets}
\end{itemize}

\subsubsection{Label}
\label{sec-3-8-2}

\url{./app/pdfs/label.rb}

\begin{verbatim}
class Label

  def initialize(gauge, size, desc, id, price, supply, updated)
    @gauge = gauge
    @size = size
    @desc = desc
    @id = id
    @price = price
    @supply = supply
    @updated = updated
  end

  attr_reader :gauge, :size, :desc, :id, :price, :supply, :updated

end
\end{verbatim}


\subsection{{\bfseries\sffamily TODO} }
\label{sec-3-9}

\begin{itemize}
\item $\square$ Testing
\item $\square$ sidekiq
\begin{itemize}
\item $\square$ background processes for creating pdfs?
\end{itemize}
\item $\boxminus$ requirements
\begin{itemize}
\item $\boxtimes$ roo
\item $\boxtimes$ chronic
\item $\square$ \ref{sec-3-3-2-1}
\end{itemize}
\item $\boxtimes$ migrate code from cj-parser
\item $\square$ user authentication
\end{itemize}



\begin{itemize}
\item $\square$ file upload
\begin{itemize}
\item $\square$ only xlsx file?
\item $\square$ AWS
\end{itemize}
\item $\square$ file storage
\begin{itemize}
\item $\square$ archival api?
\end{itemize}
\item $\square$ production
\begin{itemize}
\item $\square$ heroku
\begin{itemize}
\item $\square$ secrets
\end{itemize}
\end{itemize}
\end{itemize}
% Emacs 24.5.1 (Org mode 8.2.10)
\end{document}
